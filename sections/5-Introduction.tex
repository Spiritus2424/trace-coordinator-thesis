% Dans l'introduction, on présente le problème étudié et les buts
% poursuivis. L'introduction permet de faire connaître le cadre de la
% recherche et d'en préciser le domaine d'application. Elle fournit
% les précisions nécessaires en ce qui concerne le contexte de
% réalisation de la recherche, l'approche envisagée, l'évolution de
% la réalisation. En fait, l'introduction présente au lecteur ce
% qu'il doit savoir pour comprendre la recherche et en connaître la
% portée.
\Chapter{INTRODUCTION}\label{sec:Introduction}  % 10-12 lignes pour introduire le sujet.


\section{Définitions et concepts de base}  % environ 2-3 pages

\subsection{Multi-Thread Application}

\subsection{Distributed System Application}

\subsection{Tracing}

\subsection{Distributed Tracing}

\subsection{Trace Analysis}

\subsection{Trace Visualisation}

\subsection{Critical Path}

\subsection{Distributed Trace Analysis}

\subsection{Trace Server Protocol (TSP)}

\clearpage

%%
%% ELEMENTS DE LA PROBLEMATIQUE
%%
\section{Elements of the Problem}  % environ 3 pages

\subsection{Distributed Analysis}

\subsection{Performance of the Trace Visualisation Tools}

\subsection{Having a Standard?}

\subsection{Infrastructure Deployment}




%%
%% OBJECTIFS DE RECHERCHE / RESEARCH OBJECTIVES
%%
\section{Research Objectif}  % 0.5 page


%%
%% PLAN DU MEMOIRE / THESIS OUTLINE
%%
\section{Thesis Outline}  % 0.5 page
